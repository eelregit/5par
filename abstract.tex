\paragraph{\large Abstract}  % 200 words

The standard cosmological model, with its six independent parameters,
has proven remarkably successful in describing the evolution of the
Universe.
One of these parameters, the optical depth to reionization $\tau_\reio$,
represents the scatterings that Cosmic Microwave Background (CMB)
photons will experience after decoupling from the primordial plasma as
the intergalactic medium transitions from neutral to ionized.
$\tau_\reio$ depends on the neutral hydrogen fraction $x_\HI$, which, in
turn, should theoretically depend on cosmology.
We present a novel method to establish the missing link between
cosmology and reionization timeline.
We discover the timeline has a universal shape well described by the
Gompertz mortality law, applicable to any cosmology within our simulated
data.
This enables us to map cosmology to reionization using
symbolic regression and to treat $\tau_\reio$ as a derived parameter.
Reanalyzing CMB data with our universal $x_\HI$ tightens the constraint
on $\tau_\reio$ by more than one order of magnitude to $\approx 1\%$ and
reduced the error on the amplitude of the primordial fluctuations by a
factor of 2.5 compared to Planck's PR3 constraint.
While our results rely on the astrophysical assumptions in our
simulations, the methodology itself is independent of these assumptions;
after all, $\tau_\reio$ is fundamentally a function of cosmology.
