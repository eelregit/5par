\paragraph{\large Abstract}  % 200 words max (approx. 198 now)
The standard cosmological model, with its six independent parameters,
successfully describes the evolution of the
Universe.
One parameter, the optical depth to reionization $\tau_\reio$,
represents the scatterings that Cosmic Microwave Background (CMB)
photons will experience after decoupling from the primordial plasma as
the intergalactic medium transitions from neutral to ionized.
$\tau_\reio$ depends on the neutral hydrogen fraction $x_\HI$, which, in
turn, should theoretically depend on cosmology.
We present a novel method to establish the missing link between
cosmology and reionization timeline.
We discover the timeline has a universal shape well described by the
Gompertz mortality law, applicable to any cosmology within our simulated
data.
This enables us to map cosmology to reionization using
symbolic regression. By combining CMB with astrophysical data
and marginalizing over astrophysics, we treat $\tau_\reio$ as a derived parameter, 
tightening its constraint to $<3\%$. This approach 
reduces the error on the amplitude of the primordial fluctuations by a
factor of 2.3 compared to Planck's PR3 constraint and provides a commanding
constraint on the ionization efficiency $\zetaUV = 26.9^{+2.1}_{-2.5}$.
While our results rely on the astrophysical assumptions in our
simulations, the methodology itself is independent of these assumptions;
after all, $\tau_\reio$ is fundamentally a function of cosmology.
