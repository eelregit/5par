The $\Lambda$CDM cosmological model has proven extremely effective in
predicting the evolution of our Universe, relying on only six parameters
\cite{Planck2020a}.
In particular, it explains the transition from a predominantly neutral
state in the early stages to the familiar ionized intergalactic medium
(IGM) observed in our relatively nearby surroundings.
This transition is known as cosmic reionization.
Despite a comprehensive understanding of the astrophysical principles
governing this transition, uncertainties persist regarding its precise
timeline \cite{Jin2023}.
The advent of the James Webb Space Telescope (JWST) \cite{Gardner2006}
represents a pivotal moment, substantially bolstering our ability to
directly constrain the evolution of the neutral hydrogen fraction
$x_\HI$.
This progress is being driven by JWST's enhanced detection
capabilities, enabling the observation of high-redshift quasars
\cite{Eilers2023} and high-redshift galaxies
\cite{Adams2023, Bradley2023, Donnan2023,Ning2024}.

Reionization leads to scattering of Cosmic Microwave Background (CMB)
photons by free electrons, disrupting the CMB angular power spectra
($C_\ell$).
This scattering suppresses the signal at scales smaller than the Hubble
scale at reionization (approximately $\ell>10$) \cite{Planck2020b} due
to the optical depth $\tau_\reio$.
Additionally, it introduces a new signal in the polarization of CMB
photons at large angular scales \cite{Planck2020a}, that is $\propto
\tau_\reio$ in $C^{TE}_\ell$, the cross-correlation of the $E$-mode
polarization with the temperature (intensity), and is $\propto
\tau_\reio^2$ in $C^{EE}_\ell$, the $E$-mode polarization angular auto
power spectrum.
Consequently, heightened sensitivity to CMB polarization becomes crucial
for mitigating the degeneracy between $\tau_\reio$ and other
cosmological parameters, particularly $\As$, the amplitude of the
primordial scalar power spectrum, and $r$, the ratio of tensor-to-scalar
modes \cite{Natale2020}.

Low-$\ell$ polarization data is crucial to determine $\tau_\reio$; however,
the measurement of such a weak signal ($\sim 10^{-2} \mu$K$^2$) demands
superb systematic and foreground control\cite{Planck2020b}. Furthermore,
anomalous measurements in $C^{TE}_\ell$ at low multipoles\cite{Planck2020a}
could indicate concerns to the cosmological interpretations at these angular scales.
Ultimately, this challenging measurement may require adopting a comprehensive
Bayesian framework to jointly consider cosmology, astrophysics, and
instrument systematics\cite{Paradiso2023}. \Cref{fig:tau} illustrates current
representative constraints on $\tau_\reio$.

\begin{figure}[tb]
\centering
\includegraphics{figs/tau_fig.pdf}
\caption{\textbf{Current constraints on the optical depth to
reionization ($\tau_\reio$) from Cosmic Microwave Background (CMB)
data.}
The error bars indicate the 1$\sigma$ uncertainties.
Various analyses may employ distinct data sets or vary in the parameters
considered.
For instance, the inclusion of astrophysical data \cite{Qin2020,Paoletti2024} 
(cross and purple hexagon), WMAP data in Refs.~ \cite{Natale2020,
Paradiso2023} (circle and square), or ACT in combination with other
external data sets \cite{Giare2023} (triangle), expanded sky coverage
\cite{Paradiso2023} (square), incorporation of high-$\ell$ data
\cite{Pagano2020, Planck2020a, HeinrichHu2021, Giare2023, Tristram2024} 
(pentagon, diamond, star, triangle, and octagon), joint low-$\ell$ TT and EE 
analysis \cite{deBelsunce2021} (cyan hexagon), marginalization over 
small set of strongly correlated parameters \cite{Natale2020} (circle), 
and the implementation of an end-to-end Bayesian framework that marginalizes 
over astrophysics and instrumental systematics \cite{Paradiso2023} (square).}
\label{fig:tau}
\end{figure}

Given the challenges posed by $\tau_\reio$ in CMB analyses and the
anticipated advancements in constraining the reionization 
timeline\cite{Montero2021, Hera2022}, now is an opportune
moment to reassess its role.
Theoretically, cosmic reionization is uniquely determined by
cosmology, i.e.\  $x_\HI(z)$ is fully determined by the other five cosmological
parameters. However, incomplete understanding of
reionization obscures this mapping, necessitating the introduction
of $\tau_\reio$ in CMB analyses. Since the inclusion of 
$\tau_\reio$ became a standard practice, 
our understanding of the astrophysical 
processes governing reionization has significantly 
improved\cite{Gnedin2022, Kannan2022,Murray2020, Fan2023}  
and ongoing and forthcoming observations promise to reduce inherent
modeling uncertainties.

Motivated by these developments, we use symbolic regression 
(SR)\cite{Cranmer2023} to construct a mapping between cosmology, 
astrophysics, and reionization timeline, aiming to demote $\tau_\reio$ 
from an independent to a derived cosmological parameter 
and simultaneously tightening constraints 
in cosmological parameters. This mapping can also
shed light on reionization astrophysics and aid ongoing efforts in 
parametrizing reionization models\cite{Trac2018,Trac2022,Paoletti2024}
by including the cosmological dependence of $x_\HI$.

Here, we present a universality in the neutral hydrogen time evolution,
and derive through SR its cosmological dependence from simulated
\texttt{21cmFAST}\cite{MesingerEtAl2011,Murray2020} reionization histories.
We integrate this shape into \texttt{CLASS} \cite{Blas2011}, a popular
Boltzmann solver for CMB analyses.
We then evaluate the modified \texttt{CLASS} alongside \texttt{Cobaya}
\cite{Torrado2020}, a speed-aware sampler\cite{Lewis2002,
Lewis2013}\footnote{With adaptive covariance learning and fast-dragging
as in \cite{Neal2005}, enabling larger steps in slow parameters via
intermediate transitions of fast parameters.}, showcasing its ability to
recover parameter constraints from CMB data, including `TTTEEE' +
lensing likelihoods\cite{Planck2020c, Planck2020d} (see \Cref{fig:tg}).
Finally, combining CMB and astrophysical data, we marginalize over 
reionization astrophysics to compute $\tau_\reio$ as
a derived parameter using SR, quantifying the gains
compared to sampling over $\tau_\reio$ utilizing the conventional $\tanh$
model\cite{Lewis2008}.
We summarize our strategy in \Cref{fig:big}.

\begin{figure}[tb]
\centering
\includegraphics[width=\linewidth]{figs/big_fig.pdf}
\caption{\textbf{Strategy to demote $\tau_\reio$ to derived parameter.}
\emph{a)} Sobol sampling of $\vtheta$ comprising 5 cosmological and 1
astrophysical parameters (see \Cref{fig:sobol} in Extended Data).
\emph{b)} Simulated $x_\HI$ timelines as a function of $\vtheta$ and
scale factor $a$.
\emph{c)} With symbolic regression, we optimize the mapping from
$\vtheta$ to the rescaling parameters that bring the universality.
\emph{d)} We model the universal shape (upper panel) as a composition of
the Gompertz function and a low-degree polynomial (lower panel).
\emph{e)} Planck CMB data and $x_\HI$ data we analyze.
\emph{f)} We infer the parameter constraints using Monte Carlo Markov
Chain (MCMC).}
\label{fig:big}
\end{figure}


We construct a universal shape for $x_\HI$ using 512 Sobol 
samples of \texttt{21cmFAST} simulations. 
The simulations account for variations in the ionization efficiency ($\zetaUV$), 
which modulates the timing of reionization by regulating the abundance
of photons that escape into the IGM 
(see \nameref{ssec:sims} and \Cref{fig:sobol} in the Extended Data). 
All $x_\HI(a)$ profiles share this shape, with differences between
scenarios being mere translations and rescalings.
Reionization causes $x_\HI$ to reduce from near 1 to effectively 0 via a
sigmoid transition.
The standard $\tanh$ function is symmetric in nature and not flexible
enough to provide the early start and rapid completion suggested by
reionization simulations \cite{Trac2018, Doussot2019}.
The Gompertz curve, an asymmetric sigmoid function often used to analyze
age-dependent human mortality \cite{Gompertz1825}, proves a good model
for the survival of neutral hydrogen too.
Its expected accelerated increase in mortality with age resembles the
expectation for the percolation of ionized hydrogen bubbles during the end
stages of reionization.

One way to uncover the universality is to view each $x_\HI$ scenario as
a cumulative probability distribution (CDF) in $- \ln a$.
With this insight, we can translate and rescale each timeline using the
mean and variance of its corresponding probability density function
(PDF), i.e.\ $- \mathrm{d}x_\HI / \mathrm{d}\ln a$, and discover the
existence of a universal shape followed by all scenarios.
Therefore, cosmology -- and $\zetaUV$ -- only impacts 
the translation and rescaling parameters of each timeline, not its shape.

However, with the PDF trick, some $x_\HI$'s can deviate artificially
from universality, due to their incomplete reionization given our broad
range of simulated scenarios.
To address this, we adopt a better approach to jointly fit the global
shape and the 2 individual parameters of each $x_\HI$.
Our shape model constitutes the Gompertz function composed with a
5th-degree polynomial, in the translated-and-rescaled time $\ln\ar
\equiv \tilt (\ln a - \ln\ap)$.
And we are free to set the polynomial constant to 0 and its linear
coefficient to 1 by utilizing their respective degeneracy with $\ln\ap$
and $\tilt$.

The complete model parametrizes the HI evolution as follows (also see
\Cref{fig:shape} and \nameref{ssec:shape} in the Extended Data):
%
\begin{align}
\label{eq:uni}
x_\HI(\ar) &= \gomp\bigl( P_5(\ar) \bigr)
  \equiv \exp\bigl[ - \exp\bigl( P_5(\ar) \bigr) \bigr], \\
%
\label{eq:poly}
P_5(\ar) &= {\textstyle\sum}_{m=0}^5 \, c_m \ln^m\!\ar, \\
%
\bm{c} &= \{0, 1, 0.1130, 0.02600, 0.0005491, -0.00006518\}, \nonumber\\
%
\label{eq:map}
\ar(a; \vtheta) &= \Bigl[ \frac{a}{\ap(\vtheta)} \Bigr]^{\tilt(\vtheta)},
\end{align}
%
where $\vtheta$ denotes 6 astrophysical and cosmological parameters,
$\ap(\vtheta)$ is the power-law pivot (or logarithmic translation), and
$\tilt(\vtheta)$ is the rescaling tilt.
Their cosmological -- and astrophysical-- dependence may stem from \texttt{21cmFAST}'s
treatment of reionization astrophysics.
See \nameref{ssec:helium} for specifics on implementing HeI and HeII
reionization.

Before fully leveraging our formalism to extract the cosmological
dependence in the rescaling of \cref{eq:uni} and relaxing the
need for $\tau_\reio$ in CMB analyses, we first implement the Gompertz
shape with independent $\tau_\reio$ in \texttt{CLASS} and confirm its
agreement with the conventional $\tanh$ model (gomp and $\tanh$  in \Cref{tab:uber-table}).
Using Planck PR3 likelihoods `TTTEEE' \cite{Planck2020c} and CMB
lensing \cite{Planck2020d}, we sample typical cosmological parameters
with \texttt{Cobaya} \cite{Torrado2020}, including $\tau_\reio$. Given a proposal for
$\tau_\reio$, we determine the corresponding reionization timeline using bisection by varying
$\ln\ap$ for gomp, while for $\tanh$,
the reionization midpoint $z_\re$ is the tuning parameter.
The sampler runs until the Gelman-Rubin statistic \cite{Gelman1992}
satisfies $R - 1 < 0.2$ for the between-chain variance of the confidence
intervals.
We repeat this for $\tanh$ and verify the agreement between the two models.

\Cref{fig:tg,tab:uber-table} in the Extended Data summarize the validating
experiment.
The only notable differences in inferred parameters are in $z_\re$.
The gomp scenario suggests a more delayed reionization by over
$1\sigma$, with $z_\re = 6.76 \pm 0.67$ compared to $7.67 \pm 0.75$ for
$\tanh$, in alignment with recent high-$z$ quasar observations
\cite{Keating2020}.
All other cosmological parameters are in good agreement with Planck's
results \cite{Planck2020a}, with biases of $\lessapprox 0.5 \%$.

Gompertz-polynomial-shaped reionization can reproduce standard CMB analyses.
Now, we move to establish the connection
between the universal shape for $x_\HI$ and the rescaling of a given
reionization scenario. We refer to this model as gomp + SRFull.
This rescaling naturally depends on cosmology.
For example, a larger density of matter $\Omegam$ results in deeper
potential wells, accelerating structure formation and increasing the
number of ultraviolet photons driving the reionization process.
We employ \texttt{PySR}, an SR package, to derive 
the rescaling in \cref{eq:map}.

While \texttt{PySR} initially guided us towards the Gompertz curve when
directly regressing $x_\HI$, the final analysis only uses it to regress
the pivot and tilt instead.
We fit their values jointly with the polynomial coefficients as
described above, and feed them as labels to the genetic algorithm to
find the best analytic expression (see \nameref{ssec:pysr} in Extended
Data for our definition of \emph{best}).
Using \texttt{PySR} we derived the following mapping 
%
\begin{align}
\label{eq:SR_a}
\ln\ap(\vtheta) &= \left(\frac{\Omegab}{\Omegam}\right)^h + (0.04835 - \sigma_8) (\ns + 0.3558 \ln[0.1123 \zetaUV])  - \Omegam - \ns, \\
\label{eq:SR_b}
\beta(\vtheta) &=  \left( \frac{0.005660^{\Omegam}}{0.6015} - \ln[\zetaUV - (\Omegam + \ns h)^{15.05} ]+ h \right) \ln{\Omegab} + \frac{h}{\sigma_8},
\end{align} 
where $\ns$, $h$, $\Omegab$, and $\Omegam$ are the tilt of the
primordial power spectrum, dimensionless Hubble constant, and present
baryon and matter density fractions, respectively.
$\sigma_8$ is the present linear rms relative density fluctuation in a
sphere of radius $8 \, h^{-1}$Mpc.

\cref{eq:map,eq:SR_a,eq:SR_b} imply that higher values of $\Omegam$, $\sigma_8$, 
and $\zetaUV$ hasten reionization by enhancing structure
formation and increasing the abundance of ionizing photons. 
Similarly, larger $\ns$ primarily expedites reionization by boosting
power on small scales, leading to more ionizing 
sources and earlier completion\cite{Montero2021}. Keep in mind that our 
\texttt{21cmFAST} simulations assume that faint galaxies
are the primary drivers of reionization. Note that the role of $\sigma_8$ 
and $\ns$ in $\beta$ slightly reduces the impact of these trends. 
Surprisingly, \cref{eq:SR_a,eq:SR_b} suggests that higher $\Omegab$ delays
reionization, likely due to more HI in the intergalactic medium requiring 
additional ionizing photons. The ratio $\Omegab/\Omegam$ in $\alpha$ 
supports this view. Moreover, increasing $h$ slightly delays reionization,
consistent with the expectation that galaxies and ionizing photons will be 
more spread out. 

We note that within the prior range of our \texttt{21cmFAST} simulations
(see \nameref{ssec:sims}) and their corresponding astrophysics of
reionization, the mapping derived from SR is not unique.
Additional details and results using an alternative mapping -- SRHalf -- are
presented in \nameref{ssec:SRHalf} of the Extended Data.
Nonetheless, our results are robust and independent of the choice of
mapping.

We implement \cref{eq:SR_a,eq:SR_b} in our Gompertz \texttt{CLASS}, which given
the cosmological and astrophysical parameters, determines the pivot and tilt values, 
and consequently the reionization history, $\tau_\reio$, 
and CMB angular power spectra.
This gomp + SRFull model eliminates the need to sample over $\tau_\reio$ (or $z_\re$),
requiring only five cosmological parameters and $\zetaUV$. The 
advantage of sampling over $\zetaUV$ instead of $\tau_\reio$ is that
powerful astrophysical constraints on $x_\HI$ can be used to marginalize
over $\zetaUV$. Moreover, thanks to SR, the model can constrain 
$\zetaUV$ using CMB data directly, a link that was not utilized 
in previous efforts\cite{Greig2017}. We use \texttt{Cobaya} 
to re-analyze the same CMB data and include astrophysical data 
(see \nameref{ssec:xHI} of the Extended Data) to effectively marginalize over $\zetaUV$.

\begin{figure}[tb]
\centering
\includegraphics[width=0.7\linewidth]{figs/gomp1dw_tanh_triangle_kill.pdf}
\caption{\textbf{Analysis of CMB and astrophysical data with reionization
as a function of cosmology and astrophysics.}
The green contours represent our results using the Gompertz reionization
model with \cref{eq:SR_a,eq:SR_b}, which eliminates the need to sample 
over any conventional reionization parameter and uses quasar damping wing
constraints on $x_\HI$ to marginalize over reionization astrophysics.
The blue contours correspond to the results obtained using the
conventional $\tanh$ model, while the relevant Planck constraints
\cite{Planck2020a} are depicted with gray lines for reference. 
The $\tanh$ model performs poorly in fitting astrophysical data; thus, 
combining $\tanh$ with astrophysical data is ill-advised.}
\label{fig:kill}
\end{figure}

\Cref{fig:kill} underscores the impact of our universally-shaped
Gompertz reionization model, tightening the optical depth to  $<3\%$
compared to $> 10\%$ with the $\tanh$ prescription.
Furthermore, the constraint on $\As$ improves dramatically since the TT
data is no longer significantly hampered by the degeneracy between $\As$
and $\tau_\reio$.
The error on $\As$ decreases by an impressive factor of 2.3 compared to
Planck's results \cite{Planck2020a}.
Overall, we recover tighter constraints across the board compared to
Planck. Notably, combining SR with CMB and astrophysical data, specifically
quasar damping wing (DW) data, 
yields a highly competitive constraint on $\zetaUV = 26.9^{+2.1}_{-2.5}$, a 
commanding improvement over previous constraints
(e.g. $\zetaUV = 28^{+52}_{-18}$)\cite{Greig2017}. See
\nameref{ssec:fits}, \Cref{fig:unleashed_gomp,tab:uber-table} in the Extended Data for details.

\begin{figure}[tb]
\centering
\includegraphics[width=0.66\linewidth]{figs/history.pdf}
\caption{\textbf{Optical depth evolution $\tau(z)$ and reionization
history $x_\HI(z)$.}
Our best-fit gomp + SRFull model (green dashed line) of Planck CMB data + quasar damping wing (DW) data is
asymmetric and differs significantly from that of the symmetric $\tanh$
model (blue lines, with dotted region representing the 1$\sigma$
errors). Note $z$ is shown in logarithmic scale of $a^{-1}$.
We also include an alternative mapping from cosmology to Gompertz
timeline -- gomp + SRHalf -- in the purple dotted line (see \nameref{ssec:SRHalf}).
The shaded regions in the upper panel correspond to the inferred range
in $\tau_\reio$ from analyzing Planck PR3 data.
Additionally, the lower panel includes observational constraints from
high-redshift quasars and galaxies (see \nameref{ssec:xHI}). 
}
\label{fig:history}
\end{figure}

Our results suggest that Planck data favors a delayed reionization
compared to other CMB-based constraints (in a $\chi^2$-sense).
Our best-fit cosmological parameters indicate a midpoint of $z_\re =
6.98$ and a duration of $\Delta z \equiv z(x_\HI = 0.05) - z(x_\HI =
0.95) \approx 560 $ Myr.
While our results align with late reionization observations, the
difference from $\tanh$ is within 1$\sigma$.
The duration of reionization, though better suited to observational
constraints compared to $\tanh$, might still be considered
somewhat rapid in the context of late reionization scenarios
\cite{Cain2021}.
\Cref{fig:history} illustrates the reionization timeline derived
from our best-fit values and the inability of symmetric $\tanh$
to fit the astrophysical constraints. 

Our findings for the timeline of reionization align with late
reionization scenarios, which are supported by high-$z$ Lyman-$\alpha$
observations \cite{Keating2020, Cain2021}.
However, recent discoveries by JWST indicate the presence of massive,
bright galaxies at early redshifts $z \sim 10$
\cite{Adams2023, Bradley2023, Donnan2023}.
The presence of these early galaxies suggests a potential preference for
brighter galaxies to drive reionization, a role that in our
\texttt{21cmFAST} simulations was attributed to a population of faint
galaxies.

Furthermore, our results are influenced by the semi-numerical
prescription employed by \texttt{21cmFAST} to ionize the IGM, which,
while efficient and swift, could bias our findings.
Moreover, our exploration within the astrophysical framework of
\texttt{21cmFAST} has been limited to varying the ionization efficiency
(see \nameref{ssec:sims} in the Extended Data).
Therefore, a more comprehensive exploration is warranted to ensure, for
instance, that the choice of rescaling does not bias cosmological 
analyses. Additionally, a valuable exercise to refine the inherent relationship
between cosmological parameters and reionization history would involve
using more realistic, albeit slower, reionization models.
One such option is to use the THESAN simulations \cite{Kannan2022},
which are hydrodynamical simulations incorporating radiative transfer.

Throughout this work we considered only DW data, but including current luminosity function 
constraints can already halve the error on $\tau_\reio$. Hence, as reionization 
constraints improve, we anticipate substantial cosmological gains with
our framework.